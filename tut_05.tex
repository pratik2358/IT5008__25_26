\documentclass{beamer}
\usepackage[utf8]{inputenc}
\usepackage[T1]{fontenc}
\usepackage[english]{babel}
\usepackage{hyperref}
\usepackage{outlines}
\usepackage{graphicx}
\usepackage{algorithm}
\usepackage{algorithmic}
\usepackage{amsmath,amssymb}
\usepackage{moresize}
\usepackage{tikz}
\usetikzlibrary{overlay-beamer-styles}
\usepackage{microtype}
\usepackage[
  backend=biber,
  style=alphabetic,
]{biblatex}
\usepackage{latexsym,multicol,booktabs,calligra}
\usepackage{listings,stackengine}
\usepackage[table,dvipsnames]{xcolor}
\usepackage{cancel}
\renewcommand{\figurename}{Figure}
\renewcommand{\algorithmname}{Algorithm}

\usepackage{SHU}
\setbeamersize{text margin left=7mm, text margin right=7mm}
\definecolor{deepblue}{rgb}{0,0,0.5}
\definecolor{deepred}{rgb}{0.6,0,0}
\definecolor{deepgreen}{rgb}{0,0.5,0}
\definecolor{halfgray}{gray}{0.55}
\definecolor{nus-orange}{RGB}{239,124,0}
\definecolor{nus-blue}{RGB}{0,61,124}

\lstset{
  language=SQL,
  basicstyle=\ttfamily\footnotesize,
  keywordstyle=\bfseries\color{nus-blue},
  emphstyle=\ttfamily\color{nus-blue},
  stringstyle=\color{deepgreen},
  numbers=left,
  numberstyle=\small\color{halfgray},
  rulesepcolor=\color{nus-orange},
  frame=shadowbox,
  showstringspaces=false,
  breaklines=true,
  breakatwhitespace=true,
  keepspaces=true,
  columns=flexible,
  upquote=true
}

\author{\href{https://pratik2358.github.io/}{Pratik Karmakar}}
\title{IT5008: Tutorial 5 — Relational Algebra}
\institute{
  School of Computing,\\
  National University of Singapore
}
\date{AY25/26 S1}

\begin{document}

\begin{frame}
  \titlepage
  \begin{figure}[htpb]
    \begin{center}
      \includegraphics[keepaspectratio, scale=0.18]{nus-logo.png}
    \end{center}
  \end{figure}
\end{frame}

% ------------------------
\section{Setup}
\begin{frame}{Scenario}
\small
Students at the \textbf{National University of Ngendipura (NUN)} buy, lend, and borrow books.\\
NUNStA commissions \emph{Apasaja Private Limited} to implement an online book exchange system that records:
\begin{itemize}\itemsep3pt
  \item Students: name, faculty, department, \textbf{email}, join year.
  \item Books: title, authors, publisher, edition, \textbf{ISBN10}, \textbf{ISBN13}.
  \item Loans: \texttt{borrowed} date, \texttt{returned} date (\texttt{NULL} if active).
\end{itemize}
Auditing preserves records of graduated students and copies with loans.\\
This tutorial uses the schema/data from “SQL: Creating and Populating Tables.”
\end{frame}

% ------------------------
\section{Questions}
\begin{frame}{Questions}
\footnotesize
\begin{enumerate}
  \item Relational Algebra
  \begin{itemize}
    \item[(a)] Find the different departments in School of Computing.
    \item[(b)] Find emails of students who borrowed/lent a book \emph{before} joining the University.
    \item[(c)] Find emails of students who borrowed but did not lend a book \emph{on} joining day.
  \end{itemize}
  \item Universal Quantification
  \begin{itemize}
    \item[(a)] Find emails and names of students who borrowed \emph{all} books authored by Adam Smith.
  \end{itemize}
\end{enumerate}
\end{frame}

% ------------------------
\section{Relational Algebra}
\begin{frame}[fragile]{1(a).\ Departments in School of Computing}
\textbf{Relational Algebra:}\\
\[
\pi_{d.department}(\sigma_{d.faculty='School\ of\ Computing'}(\rho(department, d)))
\]

\textbf{SQL Equivalent:}
\begin{lstlisting}
SELECT d.department
FROM department d
WHERE d.faculty = 'School of Computing';
\end{lstlisting}
\end{frame}

\begin{frame}[fragile]{1(b).\ Borrowed or lent before joining}
\textbf{Relational Algebra:}\\
$
\pi_{s.email}(
\sigma_{(s.email=l.borrower \lor s.email=l.owner)\wedge(l.borrowed<s.year)}
(\rho(student,s)\times\rho(loan,l)))
$

\textbf{SQL Equivalent:}
\begin{lstlisting}
SELECT s.email
FROM student s, loan l
WHERE (s.email = l.borrower OR s.email = l.owner)
  AND l.borrowed < s.year;
\end{lstlisting}
\end{frame}

\begin{frame}[fragile]{1(b).\ Alternative Forms}
\textbf{Using INNER JOIN:}
\begin{lstlisting}
SELECT s.email
FROM student s
INNER JOIN loan l
ON (s.email = l.borrower OR s.email = l.owner)
AND l.borrowed < s.year;
\end{lstlisting}

\textbf{Using UNION:}
\begin{lstlisting}
SELECT s1.email
FROM loan l1, student s1
WHERE s1.email = l1.borrower
  AND l1.borrowed < s1.year
UNION
SELECT s2.email
FROM loan l2, student s2
WHERE s2.email = l2.owner
  AND l2.borrowed < s2.year;
\end{lstlisting}
\end{frame}

\begin{frame}[fragile]{1(c).\ Borrowed but did not lend on joining day}
\textbf{Relational Algebra:}\\
$
\pi_{s1.email}(\sigma_{s1.email=l1.borrower \wedge l1.borrowed=s1.year}
(\rho(student,s1)\times\rho(loan,l1)))
-
\pi_{s2.email}(\sigma_{s2.email=l2.owner \wedge l2.borrowed=s2.year}
(\rho(student,s2)\times\rho(loan,l2)))
$

\textbf{SQL Equivalent:}
\begin{lstlisting}
SELECT s1.email
FROM loan l1, student s1
WHERE s1.email = l1.borrower
  AND l1.borrowed = s1.year
EXCEPT
SELECT s2.email
FROM loan l2, student s2
WHERE s2.email = l2.owner
  AND l2.borrowed = s2.year;
\end{lstlisting}
\end{frame}

% ------------------------
\section{Universal Quantification}

\begin{frame}[fragile]{2(a).\ Borrowed all books by Adam Smith}
\textbf{SQL Nested Query (Corrected):}
\begin{lstlisting}
SELECT s.email, s.name
FROM student s
WHERE NOT EXISTS (
  SELECT 1
  FROM book b
  WHERE b.authors = 'Adam Smith'
    AND NOT EXISTS (
      SELECT 1
      FROM loan l
      WHERE l.book = b.ISBN13
        AND l.borrower = s.email));
\end{lstlisting}
\textit{Idea:} For each student, check that there is no Adam Smith book that they have not borrowed.
\end{frame}

\begin{frame}{2(a).\ Relational Algebra Strategy}
\small
Break the problem into steps (universal quantification via set difference):

\begin{enumerate}
  \item \textbf{Find all Adam Smith books.}  
    \[
      Q1 = \pi_{\text{ISBN13}}
           (\sigma_{\text{authors = 'Adam Smith'}}(\text{book}))
    \]

  \item \textbf{Form all possible student–book pairs (expected borrowings).}  
    \[
      Q2 = \pi_{\text{email, name, ISBN13}}(\text{student} \times Q1)
    \]

  \item \textbf{Find actual borrowings of Adam Smith books.}  
    \[
      Q3 = \pi_{\text{email, name, ISBN13}}
           (\text{student} \bowtie \text{loan} \bowtie
            \sigma_{\text{authors='Adam Smith'}}(\text{book}))
    \]

  \item \textbf{Identify missing borrowings (pairs in Q2 but not in Q3).}  
    \[
      Q4 = Q2 - Q3
    \]

  \item \textbf{Remove students in Q4 from the set of all students.}  
    \[
      Q5 = \pi_{\text{email, name}}(\text{student}) -
           \pi_{\text{email, name}}(Q4)
    \]
\end{enumerate}

\textbf{Answer:} $Q5$ = students who borrowed \emph{all} Adam Smith books.
\end{frame}

\begin{frame}[fragile]{2(a).\ SQL from Relational Algebra}
\begin{lstlisting}
SELECT s.email, s.name
FROM student s
EXCEPT
SELECT t.email, t.name
FROM (
  SELECT s1.email, s1.name, b1.ISBN13
  FROM student s1, book b1
  WHERE b1.authors = 'Adam Smith'
  EXCEPT
  SELECT s2.email, s2.name, b2.ISBN13
  FROM student s2
       JOIN loan l2 ON s2.email = l2.borrower
       JOIN book b2 ON b2.ISBN13 = l2.book
  WHERE b2.authors = 'Adam Smith'
) AS t;
\end{lstlisting}
\end{frame}

% ------------------------
\section{Guidelines}
\begin{frame}{Guidelines \& Remarks}
\small
\begin{itemize}\itemsep4pt
  \item Use relational algebra to reason about SQL queries.
  \item Remember: EXCEPT in SQL corresponds to set difference (−) in algebra.
  \item Joins in SQL correspond to natural/θ-joins in algebra.
  \item Break universal quantification into difference-based subqueries.
  \item Keep queries readable: aliases, indentation, uppercase SQL keywords.
\end{itemize}
\end{frame}

\begin{frame}
\begin{center}
Questions?\\
Drop a mail at: pratik.karmakar@u.nus.edu
\end{center}
\end{frame}

\end{document}
